\documentclass[12pt, oneside, final]{article}
\usepackage[ngerman]{babel}
\usepackage[a4paper, left=35mm,top=26mm,right=26mm,bottom=15mm]{geometry}
\usepackage{mathptmx}
\usepackage{amsmath}
\usepackage[
    backend = biber,
    style = numeric,
    sorting = nty
]{biblatex}
\addbibresource{./assets/Quellen.bib}
\usepackage{graphicx}
\setlength{\parskip}{0.5em}
\usepackage{fancyhdr}
\pagestyle{fancy}
\fancyhf{}
\fancyhead[L]{\nouppercase{\rightmark}}
\fancyhead[R]{\thepage}

\makeatletter    
\renewcommand{\sectionmark}[1]{\markboth{\thesection\ #1}{\thesection\ #1}}
\renewcommand{\subsectionmark}[1]{\markright{\thesubsection\ #1}}
\makeatother

\author{Oğuzhan Aygün, Abdul-Malik Jakupi}
\title{Projektarbeit: Optische Dünnschichtsysteme}
\date{WiSe 2025}

\begin{document}
\maketitle
\clearpage
\tableofcontents
\clearpage
\section{Das Prinzip optischer Dünnschichtsysteme}

Der Begriff optische Dünnschichtsysteme bezeichnet üblicherweise ein System aus
oft nur wenigen Nanometer dünnen Schichten, dessen Zweck darin liegt, Wellen
und ihre \textit{Interferenz} gezielt zu nutzen, um einen gewünschten
Reflexions- und Transmissionsgrad zu erreichen. Dünnschichtsysteme sind ein
wichtiger Bestandteil im Bereich der Optik und haben auch in anderen
Fachgebieten einige Anwendungen, z.B.:
\begin{itemize}
    \item Antireflexbeschichtungen in Brillengläsern, Hochreflexschichten wie in Spiegeln
          oder Kameraobjektiven in der Optik
    \item Beschichtung von Geräten in der Medizintechnik
    \item Herstellung von Prozessoren bei Mikroelektronik
\end{itemize}
Dafür bedient man sich meist verschiedenster Materialien wie
z.B.Titanoxid, Magnesiumfluorid oder Aluminiumoxid, die alle einen
individuellen Brechungsindex besitzen.
Diese Eigenschaft werden in allen Anwendungen als das essentielle
Werkzeug genutzt, um das jeweilige Ziel im gewünschten Wellenbereich zu erreichen.
Dies kann z.B. im Fall von Brillengläsern destruktive Interferenz sein, bei anderen
Anwendungen, wie Spiegeln, aber auch konstruktive Interferenz.

Im Folgenden wird die Berechnung des bereits besprochenen Reflexions- und
Transmissionsgrades beschrieben, um ein möglichst realistisches Programm zur
Erfassung eben dieser zu entwickeln. Sie wird in Abhägigkeit der vorliegenden
Brechungsindize, dicke der Schichten und dem Einfallswinkel berechnet.

Das größte Problem, das es hier zu lösen gilt, ist die Berechnung des
Reflexions- und Transmissionsgrades, da es durch die Existenz mehrerer
Schichten, zu einem endlosen Prozess von Rekalkulationen durch Teilwellen
kommt, bei der bei jedem Durchdringen einer der Schichten ein neuer Prozess in
beide Richtungen gestartet wird, was die Berechnung auf üblichem Wege
erschwert. Eine Visualisierung des Prozesses in einem Mehrschichtsystem ist in
Abbildung 1 dargestellt. Um dieses Problem zu lösen wird die
Transfermatrixmethode angewandt, welche anhand der Brechungsindizes, des
Einfallswinkels und den Reflexions- und Transmissionskoeffizienten, an jeder
Grenzfläche, ein Reflexionspektrum als Resultat liefert.

\begin{figure}[b]
    \centering
    \includegraphics[scale=0.2]{./assets/Grafik Reflexion.png}
    \caption{Darstellung der Reflexion und Transmission im Mehrschichtsystem}
    \label{fig:Reflexion}
\end{figure}

\section{Mathematisch-Physikalisches Modell}
\subsection{Brechungsindex}
Der (komplexe) Brechungsindex der jeweiligen Schicht hat die Form
\[\tilde{n} = n + i\kappa.\]
Sein realer Teil bestimmt die Brechung des Lichts an der Grenzfläche, während
sein imaginärer Teil die Absorption des Lichts beschreibt. Die
Wellenausbreitung kann durch den Term
\[e^{i\frac{n \omega}{c}z} \cdot e^{-\frac{\kappa\omega}{c}z}.\]
beschrieben werden. Dabei ist die Absorption
\[e^{-\frac{\kappa\omega}{c}z}\]
und die Phasenverschiebung
\[e^{i\frac{n \omega}{c}z}.\]
Diese beiden Komponenenten sind essentiell für spätere Berechnungen von
Schichteigenschaft mit der Transfermatrixmethode.

Des Weiteren ist es wichtig den Brechungsindex als Funktion der Wellenlänge
$n(\lambda)$ zu betrachten, um präzise Resultate zu erlangen, da Dispersion
berücksichtigt werden muss. Dafür wird im Laufe der Arbeit, wenn möglich, die
Sellmeier-Gleichung verwendet\cite{10.1063/1.555616}.
\[n^2(\lambda) = 1 + \sum_{i} \frac{B_i \lambda^2}{\lambda^2 - C_i}\]
Für manche Materialien wie Gase mussten für ein präziseres Ergebnis andere
Approximationen genutzt werden\cite{Kren:11}.
\subsection{Reflexion und Transmission}
Reflexions- und Transmissionskoeffizienten von elektromagnetischen Wellen an
einer Grenzfläche zwischen zwei Schichten werden mit den Fresnelschen Formeln
ermittelt\cite{Hecht+2018}.

$\alpha$ ist der Einfallswinkel des Lichts auf die Grenzfläche und $\beta$ der
Brechungswinkel. Der Brechungswinkel lässt sich
durch das Snelliussche Brechungsgesetz definieren\cite{Hecht+2018}:
\[n_1 \sin\alpha = n_2 \sin\beta\]
Durch Auflösen nach $\beta$ entsteht folgende Gleichung:
\[\beta = \arcsin(\frac{n_1}{n_2}\sin\alpha)\]
zusätzlich muss die Polarisation berücksichtigt werden. Abhängig davon, ob das
Licht senkrecht oder parallel zur Einfallsebene schwingt, variieren die
Formeln.

Senkrechte Polarisation:
\[r_s = \frac{n_1 \cos\alpha - n_2 \cos\beta}{n_1 \cos\alpha + n_2 \cos\beta}, \quad t_s = \frac{2n_1 \cos\alpha}{n_1 \cos\alpha + n_2 \cos\beta}\]

Parallele Polarisation:
\[r_p = \frac{n_2 \cos\alpha - n_1 \cos\beta}{n_2 \cos\alpha + n_1 \cos\beta}, \quad t_p = \frac{2n_1 \cos\alpha}{n_2 \cos\alpha + n_1 \cos\beta}\]
\[\]
$n_1$ und $n_2$ stellen die Brechungsindizes der linken und rechten Schicht beim Grenzübergang da.

Dieser Prozess liefert nun die Koeffizienten an der Grenzfläche zwischen zwei
Schichten und muss für jede Schicht, bzw. jeden Grenzübergang erfolgen. Die
Resultate werden in der Transfermatrixmethode verwertet.
\subsection{Die Transfermatrixmethode}
Es handelt sich bei der Transfermatrixmethode um eine elegante mathematische
Lösung für das Problem der unendlichen Teilwellen\cite{Hecht+2018}. Anstatt den
Verlauf des Lichtstrahls zu verfolgen, wird das gesamte Mehrschichtsystem
betrachtet. Die Grundlage hierfür sind die fresnelschen Formeln, da sie das
Verhältnis der Amplituden auf der linken und rechten Seite der Grenzfläche
beschreiben. Das resultierende Gleichungssystem kann auch in Matrixschreibweise
geschrieben werden, sodass klar wird, dass durch Matrixmultiplikation in den
wie folgt definierten Matrizen, der Reflexionsgrad berechnet werden kann.

Es werden jeweils zwei Arten von Matrizen gebildet, dabei bestimmt die Matrix D
die Reflexion und Transmission an der Grenzfläche, während die Matrix P die
Phasenverschiebung innerhalb einer Schicht beschreibt. Die Matrix D besitzt die
Form
\[D = \frac{1}{t}\begin{pmatrix}
        1 & r \\
        r & 1
    \end{pmatrix},\]
wobei $r$ dem berechneten Reflexions- und $t$ dem Transmissionskoeffizienten
aus den fresnelschen Formeln entsprechen. Die Matrix P hat die Form
\[P = \begin{pmatrix}
        e^{-i \varphi} & 0             \\
        0              & e^{i \varphi}
    \end{pmatrix}\]
mit
\[\varphi = \frac{2 \pi n d \cos(\beta)}{\lambda},\]
wobei d die Dicke, n der Brechungsindex und $\beta$ der Brechungswinkel der
Schicht sind. Die Transfermatrix selbst ergibt sich nun aus der Multiplikation
der enstandenen Matrizen
\[M = D_0 \cdot P_0 \cdot D_{1} \cdot P_{1} \dots D_{n-1} \cdot P_n \cdot D_n\]
und wird anschließend genutzt um den Reflexionsgrad R, nach der Definition der
Intensität einer elektromagnetischen Welle,
\[R = \left| \frac{M_{21}}{M_{11}} \right|^2\]
zu bestimmen.
\section{Algorithmische Umsetzung}
\section{Benutzerhandbuch}
Hier wird die grafische Nutzeroberfläche der Software präsentiert, zusammen mit
einzelnen Funktionen, denen es eine Erläuterung bedarft.

Zu aller erst müssen die Abhägigkeiten installiert werden. Da die Software auf
Python basiert, kann dies am einfachsten über das Paketverwaltungsprogramm pip
getätigt werden. In der Repository ist eine \texttt{requirements.txt} dafür zu
finden, welche über pip mit dem Befehl
\begin{center}
    \texttt{pip install -r requirements.txt}
\end{center}
zu installieren sind. Anschließend wird das Programm nach erfolgreicher
Installation mit
\begin{center}
    \texttt{Python GUI.py}
\end{center}
ausgeführt.

Nun sollte eine GUI zu sehen sein, welche aus einer Tabelle, einem Graphen und
einzelnen zu bestimmenden Parametern unter beiden Flächen besteht. Dies ist
zusätzlich in Abbildung 2 zusehen.
\begin{figure}[htbp]
    \centering
    \includegraphics[scale=0.45]{./assets/GUI.png}
    \caption{Beschreibung der Benutzeroberfläche}
    \label{fig:GUI}
\end{figure}
Es stehen in der Tabelle bereits 2 Zeilen vorangefertigt zur Verfügung. Sie
stellen in der Software das Einfallsmedium und Substrat da und können
entsprechend nicht entfernt werden. Am Einfallsmedium befindet sich in der
letzten Spalte ein Button, welcher eine weitere Zeile, bzw. Schicht, direkt
nach ihr selbst generiert und einfügt. Alle Zeilen, die eine Schicht
repräsentieren, welche weder das Einfallsmedium, noch das Substrat sind,
besitzen zwei Buttons in ihrer letzten Spalte. Das Pluszeichen hat die selbe
Funktion, wie beim Einfallsmedium, während das Minuszeichen die Schicht, in der
sie beeinhaltet ist, aus der Simulation löscht.

In der ersten Spalte unter \texttt{Material} ist ein Dropdown-Menü mit der
Aufschrifft Presets zu sehen. Benutzer können hier eine Auswahl von 5
vorgefertigten Schichten wählen, welche bereits alle Parameter belegt haben. Es
steht dem Nutzer jedoch offen die Dicke der Schicht anzupassen. Alternativ
besteht auch die Möglichkeit mit dem Button unter der Tabelle ein neues
Material anzulegen. Das betätigen öffnet ein zusätzliches Fenster, in dem der
Nutzer einen Namen für das Material und einen komplexen Brechungsindex
festlegen kann. Das Bestätigen der Eingabe erstellt ein neues Materialobjekt
und fügt dessen Daten in die \texttt{Material.json} im Root ein, damit sie
zusammen mit den vorgefertigten Materialien für den Nutzer langzeitig
gespeichert sind. In Abbildung 3 ist das Dialogfenster abgebildet.
\begin{figure}[htbp]
    \centering
    \includegraphics[scale=0.6]{./assets/neu.png}
    \caption{Dialogfenster zum hinzufügen neuer Materialien}
    \label{fig:GUI-neu}
\end{figure}

\texttt{Bestätigen} plottet den Reflexionsgrad in Abhägigkeit
des angegeben Wellenlängenbereichs in den Graphen. Sollten die Parameter in
einem ungültigen Bereich liegen, wird stattdessen eine Fehlermeldung geliefert.
Weiteres Nutzen des Bestätigen-Buttons fügt die gewünschten Funktionen in den
Graph hinzu, bis der Nutzer sich entscheidet, diesen mit \texttt{Zurücksetzen}
zu leeren.

Eine Übersicht zu den Parametern und validen Eingaben befindet sich in Tabelle
1.
\begin{table}[htbp]
    \centering
    \caption{Beschreibung der Eingabeparameter im Programm}
    \label{tab:Parameter}
    \begin{tabular}{|l|p{9cm}|}
        \hline
        Parameter           & Beschreibung                                                             \\
        \hline
        Dicke               & Dicke der entsprechenden Schicht in Nanometern                           \\
        \hline
        n-real              & Realer Teil des (komplexen) Brechungsindex der
        entsprechenden Schicht                                                                         \\
        \hline
        n-imaginärer        & Imaginärer Teil des (komplexen) Brechungsindex
        der entsprechenden Schicht                                                                     \\
        \hline
        Wellenlängenbereich & Gewünschter Wellenlängenbereich $>0$, der im Graphen
        geplottet werden soll. Kann angegeben werden in der Form: \texttt{1.Wellenlänge-2.Wellenlänge} \\
        \hline
        Einfallswinkel      & Einfallswinkel in Gradmaß im Bereich $0^\circ-89^\circ$                  \\
        \hline
        Polarisation        & Art der Polarisation mit Auswahl zwischen senkrecht und parallel         \\
        \hline
    \end{tabular}
\end{table}

\section{Simulationsergebnisse}

Das Erste zu simulierende Schichtsystem ist in Tabelle 2 zu sehen. Es ist
aufgeteilt in die jeweiligen Materialien und ihren individuellen Dicken und
soll im sichtbaren Spektrum simuliert werden. Hierfür nehmen wir einen
Wellenlängenbereich von 400nm bis 700nm mit einem senkrechten Einfallswinkel.
\begin{table}[h]
    \centering
    \caption{Zu simulierendes Schichtsystem}
    \label{tab:Angabe1}
    \begin{tabular}{|l |c|}
        \hline
        Material    & Dicke in nm \\
        \hline
        Luft        & $\infty$    \\
        \hline
        MgF$_2$     & 102         \\
        TiO$_2$     & 105         \\
        AL$_2$O$_3$ & 79          \\
        \hline
        Glas        & $\infty$    \\
        \hline
    \end{tabular}
\end{table}

Das Ergbenis ist in Abbildung 4 dargestellt und zeigt keine physikalischen
Auffälligkeiten, die auf einen Fehler deuten würden. Die Funktion ist im
ausgewählten Bereich stetig, was auf mathematische Stabilität deutet.
\begin{figure}[htpb]
    \centering
    \includegraphics[scale=0.5]{./assets/1 Ergebnis Parallel 0.png}
    \caption{1. Simulationsergebnis für $R(\lambda)$ bei senkrechtem Einfall}
    \label{fig:Ergebnis1-0-p}
\end{figure}

\printbibliography[]
\end{document}