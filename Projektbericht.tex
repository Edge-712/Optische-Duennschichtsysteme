\documentclass[12pt]{article}
\usepackage[ngerman]{babel}
\usepackage[a4paper]{geometry}
\usepackage{mathptmx}
\usepackage{amsmath}
\usepackage{graphicx}
\setlength{\parskip}{0.5em}
\usepackage{fancyhdr}
\pagestyle{fancy}
\fancyhf{}
\fancyhead[L]{\nouppercase{\rightmark}}
\fancyhead[R]{\thepage}

\makeatletter    
\renewcommand{\sectionmark}[1]{\markboth{\thesection\ #1}{\thesection\ #1}}
\renewcommand{\subsectionmark}[1]{\markright{\thesubsection\ #1}}
\makeatother

\author{Oğuzhan Aygün, Abdul-Malik Jakupi}
\title{Projektarbeit: Optische Dünnschichtsysteme}
\date{WiSe 2025}

\begin{document}
\maketitle
\clearpage
\tableofcontents
\clearpage
\section{Das Prinzip optischer Dünnschichtsysteme}

Der Begriff optische Dünnschichtsysteme bezeichnet üblicherweise ein System aus
oft nur wenigen Nanometer dünnen Schichten, dessen Zweck darin liegt, Wellen
und ihre \textit{Interferenz} gezielt zu nutzen, um einen gewünschten
Reflexions- und Transmissionsgrad zu erreichen. Dünnschichtsysteme sind ein
wichtiger Bestandteil im Bereich der Optik und haben auch in anderen
Fachgebieten einige Anwendungen, z.B.:
\begin{itemize}
    \item Antireflexbeschichtungen in Brillengläsern, Hochreflexschichten wie in Spiegeln
          oder Kameraobjektiven in der Optik
    \item Beschichtung von Geräten in der Medizintechnik
    \item Herstellung von Prozessoren bei Mikroelektronik
\end{itemize}
Dafür gebraucht man sich meist an verschiedensten Materialien wie
z.B.Titanoxid, Magnesiumfluorid oder Aluminiumoxid, die alle einen
individuellen, sich voneinander unterscheidenden Brechungsindex besitzen.
Diese Eigenschaft werden in allen Anwendungen als das essentielle
Werkzeug genutzt, um das jeweilige Ziel im gewünschten Wellenbereich zu erreichen. 
Dies kann z.B. im Fall von Brillengläsern destruktive Interferenz sein, bei anderen
Anwendungen, wie Spiegeln, aber auch konstruktive Interferenz.

Im Folgenden wird die Simulation des bereits besprochenen Reflexions- und
Transmissionsgrades beschrieben, um ein möglichst realistisches Programm zur
Erfassung eben dieser zu entwickeln. Das Grundlegende Problem, dass es hier zu
lösen gilt, ist die \textit{effiziente} Berechnung des Reflexions- und
Transmissionsgrades, da es durch die Existenz mehrerer Schichten, zu einem
langen Prozess von Rekalkulationen kommt, bei der bei jedem Durchdringen einer
der Schichten ein neuer Prozess in beide Richtungen gestartet wird, was die
Komplexität dieser Berechnung um einiges erhöht. Eine Visualisierung dieses
Prozesses ist in Abbildung 1 dargestellt.

\begin{figure}[b]
    \centering
    \includegraphics[scale=0.2]{Grafik Reflexion.png}
    \caption{Darstellung der Reflexion und Transmission im Mehrschichtsystem}
    \label{fig:reflexion}
\end{figure}
\section{Mathematisch-Physikalisches Modell}
Wie bereits erwähnt müssen genau zwei Dinge bewältigt werden, um dieses Prinzip
in eine mathematische Form zu bringen. Die Reflexion- und Transmission muss
ermittelt werden und um einige zusätzliche Rechenoperationen zu vermeiden, muss
die Transfermatrixmethode genutzt werden.
\subsection{Reflexion und Transmission}
Im allgemeinen werden die Reflexions-und Transmissionskoeffizienten von
elektromagnetischen Wellen, an einer Grenzfläche zwischen zwei Schichten, mit
den \textit{fresnelschen Formeln} ermittelt.

Ein wichtiger Aspekt der mit einbezogen wird ist die Polarisation. Abhänhgig
davon ob Polarisation senkrecht oder parallel vorliegt, variieren die Formeln.

Senkrechte Polarisation:
\[r_s = \frac{n_1 \cos(\alpha) - n_2 \cos\beta}{n_1 \cos\alpha + n_2 \cos\beta}\]

Parallele Polarisation:
\[r_p = \frac{n_2 \cos(\alpha) - n_1 \cos\beta}{n_2 \cos\alpha + n_1 \cos\beta}\]

\subsection{Die Transfermatrixmethode}
\clearpage
\section{Algorithmische Umsetzung}
\clearpage
\section{Benutzeroberfläche}
\clearpage
\section{Simulationsergebnisse}

\end{document}