\documentclass[12pt, oneside, final]{article}
\usepackage[ngerman]{babel}
\usepackage[a4paper, left=35mm,top=26mm,right=26mm,bottom=15mm]{geometry}
\usepackage{mathptmx}
\usepackage{amsmath}
\usepackage{graphicx}
\setlength{\parskip}{0.5em}
\usepackage{fancyhdr}
\pagestyle{fancy}
\fancyhf{}
\fancyhead[L]{\nouppercase{\rightmark}}
\fancyhead[R]{\thepage}

\makeatletter    
\renewcommand{\sectionmark}[1]{\markboth{\thesection\ #1}{\thesection\ #1}}
\renewcommand{\subsectionmark}[1]{\markright{\thesubsection\ #1}}
\makeatother

\author{Oğuzhan Aygün, Abdul-Malik Jakupi}
\title{Projektarbeit: Optische Dünnschichtsysteme}
\date{WiSe 2025}

\begin{document}
\maketitle
\clearpage
\tableofcontents
\clearpage
\section{Das Prinzip optischer Dünnschichtsysteme}

Der Begriff optische Dünnschichtsysteme bezeichnet üblicherweise ein System aus
oft nur wenigen Nanometer dünnen Schichten, dessen Zweck darin liegt, Wellen
und ihre \textit{Interferenz} gezielt zu nutzen, um einen gewünschten
Reflexions- und Transmissionsgrad zu erreichen. Dünnschichtsysteme sind ein
wichtiger Bestandteil im Bereich der Optik und haben auch in anderen
Fachgebieten einige Anwendungen, z.B.:
\begin{itemize}
    \item Antireflexbeschichtungen in Brillengläsern, Hochreflexschichten wie in Spiegeln
          oder Kameraobjektiven in der Optik
    \item Beschichtung von Geräten in der Medizintechnik
    \item Herstellung von Prozessoren bei Mikroelektronik
\end{itemize}
Dafür bedient man sich meist verschiedenster Materialien wie
z.B.Titanoxid, Magnesiumfluorid oder Aluminiumoxid, die alle einen
individuellen Brechungsindex besitzen.
Diese Eigenschaft werden in allen Anwendungen als das essentielle
Werkzeug genutzt, um das jeweilige Ziel im gewünschten Wellenbereich zu erreichen.
Dies kann z.B. im Fall von Brillengläsern destruktive Interferenz sein, bei anderen
Anwendungen, wie Spiegeln, aber auch konstruktive Interferenz.

Im Folgenden wird die Berechnung des bereits besprochenen Reflexions- und
Transmissionsgrades beschrieben, um ein möglichst realistisches Programm zur
Erfassung eben dieser zu entwickeln. Sie wird in Abhägigkeit der vorliegenden
Brechungsindize, dicke der Schichten und dem Einfallswinkel berechnet.

Das grundlegende Problem, das es hier zu lösen gilt, ist die
\textit{effiziente} Berechnung des Reflexions- und Transmissionsgrades, da es
durch die Existenz mehrerer Schichten, zu einem langen Prozess von
Rekalkulationen kommt, bei der bei jedem Durchdringen einer der Schichten ein
neuer Prozess in beide Richtungen gestartet wird, was die Komplexität dieser
Berechnung um einiges erhöht. Eine Visualisierung des Prozesses in einem
Mehrschichtsystem ist in Abbildung 1 dargestellt. Genutzt wird dafür die
Transfermatrixmethode, welche anhand der Polarisation, des Einfallswinkels und
den Reflexions- und Transmissionskoeffizienten, an jeder Grenzfläche, ein
Reflexionspektrum als Resultat liefert.

\begin{figure}[b]
    \centering
    \includegraphics[scale=0.2]{Grafik Reflexion.png}
    \caption{Darstellung der Reflexion und Transmission im Mehrschichtsystem}
    \label{fig:reflexion}
\end{figure}

\section{Mathematisch-Physikalisches Modell}
Wie bereits erwähnt müssen genau zwei Dinge bewältigt werden, um dieses Prinzip
in eine mathematische Form zu bringen. Die Reflexion- und Transmission muss
ermittelt werden und um einige zusätzliche Rechenoperationen zu vermeiden, muss
die Transfermatrixmethode genutzt werden.
\subsection{Reflexion und Transmission}
Im Allgemeinen werden die Reflexions-und Transmissionskoeffizienten von
elektromagnetischen Wellen, an einer Grenzfläche zwischen zwei Schichten, mit
den \textit{fresnelschen Formeln} ermittelt. Dabei ist $n$ der (komplexe)
Brechungsindex der jeweiligen Schicht. Ihr reeler Teil ist verantwortlich für
die Darstellung der Brechung des Lichts an der Grenzfläche, während ihr
imaginärer Teil die Absorption des Lichts beschreibt. Sie kann durch den Term
\[e^{i\frac{(n + i\kappa)\varphi}{c}z}\]
formuliert werden, aber um einen Term zu erhalten, der jeweils den reelen- und
den komplexen Brechungsindex beschreibt, wird wie folgt umgestellt:
\[e^{i\frac{n \varphi}{c}z} \cdot e^{-\frac{\kappa\varphi}{c}z}.\]
Dabei handelt es sich um die Absorption
\[e^{-\frac{\kappa\varphi}{c}z}\]
und die Phasenverschiebung
\[e^{i\frac{n \varphi}{c}z}.\]
Das $\alpha$ ist der Einfallswinkel in die Grenzfläche und $\beta$ der
Brechungswinkel zwischen den beiden Schichten. Der Brechungswinkel lässt sich
durch das Snelliussche Brechungsgesetz definieren:
\[n_1 \sin\alpha = n_2 \sin\beta\]
Wird nun schlicht nach $\beta$ aufgelöst entsteht folgende Definition:
\[\beta = \arcsin(\frac{n_1}{n_2}\sin\alpha)\]
Der letzte wichtige Aspekt der mit einbezogen wird ist die Polarisation.
Abhängig davon ob das Licht senkrecht oder parallel zur Einfallsebene schwingt,
variieren die Formeln.

Senkrechte Polarisation:
\[r_s = \frac{n_1 \cos\alpha - n_2 \cos\beta}{n_1 \cos\alpha + n_2 \cos\beta}, \quad t_s = \frac{2n_1 \cos\alpha}{n_1 \cos\alpha + n_2 \cos\beta}\]

Parallele Polarisation:
\[r_p = \frac{n_2 \cos\alpha - n_1 \cos\beta}{n_2 \cos\alpha + n_1 \cos\beta}, \quad t_p = \frac{2n_1 \cos\alpha}{n_2 \cos\alpha + n_1 \cos\beta}\]
\[\]

Dieser Prozess liefert nun die Koeffizienten an der Grenzfläche zwischen zwei
Schichten und muss für jede Schicht, bzw. Grenzüberschreitung in die nächste
Schicht erfolgen. Die Resultate werden in der Transfermatrixmethode verwertet.
\subsection{Die Transfermatrixmethode}
Es handelt sich bei der Transfermatrixmethode um eine elegante mathematische
Lösung für das Problem der unendlichen Teilwellen. Anstatt schlicht den Verlauf
des Lichtstrahls zu verfolgen, wird das gesamte Mehrschichtsystem betrachtet.
Die Grundlage hierfür sind die fresnelschen Formeln, da sie das Verhältnis der
Amplituden auf der linken und rechten Seite der Grenzfläche beschreiben. Das
resultierende Gleichungssystem kann auch in Matrixschreibweise geschrieben
werden, sodass klar wird, dass durch Matrixmultiplikation in den wie folgt
definierten Matrizen, der Reflexionsgrad berechnet werden kann.

Es werden jeweils zwei Arten von Matrizen gebildet, dabei bestimmt die Matrix D
die Reflexion und Transmission an der Grenzfläche, während die Matrix P die
Phasenverschiebung innerhalb einer Schicht beschreibt. Die Matrix D besitzt die
Form
\[D = \frac{1}{t}\begin{pmatrix}
        1 & r \\
        r & 1
    \end{pmatrix},\]
wobei $r$ dem berechneten Reflexions- und $t$ dem Transmissionskoeffizienten
aus den fresnelschen Formeln entsprechen. Die Matrix P hat die Form
\[P = \begin{pmatrix}
        e^{-i \varphi} & 0             \\
        0              & e^{i \varphi}
    \end{pmatrix}\]
mit
\[\varphi = \frac{2 \pi n d \cos(\beta)}{\lambda},\]
wobei d die Dicke, n der Brechungsindex und $\beta$ der Brechungswinkel der
Schicht sind. Die Transfermatrix selbst ergibt sich nun aus der Multiplikation
der enstandenen Matrizen
\[M = D_0 \cdot P_0 \cdot D_{1} \cdot P_{1} \dots D_{n-1} \cdot P_n \cdot D_n\]
und wird anschließend genutzt um den Reflexionsgrad R, nach der Definition der
Intensität einer elektromagnetischen Welle,
\[R = \left| \frac{M_{21}}{M_{11}} \right|^2\]
zu bestimmen.
\section{Algorithmische Umsetzung}
\section{Benutzeroberfläche}
Hier wird die grafische Nutzeroberfläche der Software präsentiert, zusammen mit
einzelnen Funktionen, denen es eine Erläuterung bedarft.

Zu aller erst müssen die Abhägigkeiten installiert werden. Da die Software auf
Python basiert, kann dies am einfachsten über das Paketverwaltungsprogramm pip
getätigt werden. In der Repository ist eine \texttt{requirements.txt} dafür zu
finden, welche über pip mit dem Befehl
\begin{center}
    \texttt{pip install -r requirements.txt}
\end{center}
zu installieren sind. Anschließend wird das Programm nach erfolgreicher
Installation mit
\begin{center}
    \texttt{Python GUI.py}
\end{center}
ausgeführt.

Nun sollte eine GUI zu sehen sein, welche aus einer Tabelle, einem Graphen und
einzelnen zu bestimmenden Parametern unter beiden Flächen besteht.

\section{Simulationsergebnisse}

\end{document}