\documentclass[12pt]{article}
\usepackage[ngerman]{babel}
\usepackage[a4paper]{geometry}
\usepackage{mathptmx}
\usepackage{amsmath}
\setlength{\parskip}{0.5em}
\usepackage{fancyhdr}
\pagestyle{fancy}
\fancyhf{}
\fancyhead[L]{\nouppercase{\rightmark}}
\fancyfoot[R]{\thepage}

\makeatletter
\renewcommand{\sectionmark}[1]{\markboth{\thesection\ #1}{\thesection\ #1}}
\renewcommand{\subsectionmark}[1]{\markright{\thesubsection\ #1}}
\makeatother

\author{Abdul-Malik Jakupi, Oğuzhan Aygün}
\title{Projektarbeit: Optische Dünnschichtsysteme}
\date{WiSe 2025}

\begin{document}
\maketitle

\break
\tableofcontents

\break

\section{Das Prinzip optischer Dünnschichtsysteme}

Der Begriff optische Dünnschichtsysteme bezeichnet üblicherweise ein System an
oft nur wenigen Nanometer dünnen Schichten, dessen Zweck darin liegt, Wellen
und ihre \textit{Interferenz} gezielt zu nutzen, um einen gewünschten
Reflexions- und Transmissionsgrad zu erreichen. Dünnschichtsysteme sind ein
wichtiger Bestandteil im Bereich der Optik, da sie einige Anwendungen in
verschiedenen Fachgebieten haben, wie z.B.:

\begin{itemize}
    \item Antireflexbeschichtungen in Brillengläsern, Hochreflexschichten wie in Spiegeln
          oder Kameraobjektiven in der Optik
    \item Beschichtung von Geräten in der Medizintechnik
    \item Herstellung von Prozessoren bei Mikroelektronik
\end{itemize}
Dafür gebraucht man sich meist an verschiedensten Materialien wie
z.B.Titanoxid, Magnesiumfluorid oder Aluminiumoxid, von welchen alle einen
individuellen, sich voneinander unterscheidenden, Brechungsindex besitzen.
Diese Eigenschaft nutzen wir in allen Anwendungen als unser essentiellstes
Werkzeug, um unser jeweiliges Ziel im gewünschten Wellenbereich zu erreichen.

Im Folgenden haben wir uns mit der Simulation des bereits besprochenen
Reflektions- und Transmissionsgrades befasst, um ein möglichst realistisches
Programm zur Erfassung eben dieser zu entwickeln. Das Grundlegende Problem,
dass es hier zu lösen gilt, ist die \textit{effiziente} Berechnung des
Reflektions- und Transmissionsgrades, da es durch die Existenz mehrerer
Schichten, zu einem sehr langen Prozess von Rekalkulationen kommt, bei der bei
jedem Durchdringen einer der Schichten ein neuer Prozess in beide Richtungen
gestartet wird, was die Komplexität unserer Berechnung um einiges erhöht.

\break
\section{Mathematisch-Physikalisches Modell}
\subsection{Reflexion und Transmission}

\break
\subsection{Die Transfermatrixmethode}

\break
\section{Algorithmische Umsetzung}

\break
\section{Benutzeroberfläche}

\break
\section{Simulationsergebnisse}

\end{document}