\documentclass[12pt, oneside, final]{article}
\usepackage[ngerman]{babel}
\usepackage[T1]{fontenc} % richtige Silbentrennung || nicht löschen!
\usepackage[a4paper, left=35mm,top=26mm,right=26mm,bottom=15mm]{geometry}
\usepackage{mathptmx}
\usepackage{amsmath}
\usepackage[
    backend = biber,
    style = numeric,
    sorting = nty
]{biblatex}
\addbibresource{./assets/Quellen.bib}
\usepackage{graphicx}
\setlength{\parskip}{0.5em}
\usepackage{fancyhdr}
\pagestyle{fancy}
\fancyhf{}
\fancyhead[L]{\nouppercase{\rightmark}}
\fancyhead[R]{\thepage}

\makeatletter    
\renewcommand{\sectionmark}[1]{\markboth{\thesection\ #1}{\thesection\ #1}}
\renewcommand{\subsectionmark}[1]{\markright{\thesubsection\ #1}}
\makeatother

\author{Oğuzhan Aygün, Abdul-Malik Jakupi}
\title{Projektarbeit: Optische Dünnschichtsysteme}
\date{WiSe 2025}

\begin{document}
\maketitle
\clearpage
\tableofcontents
\clearpage
\section{Das Prinzip optischer Dünnschichtsysteme}

Der Begriff optische Dünnschichtsysteme bezeichnet üblicherweise ein System aus
oft nur wenigen Nanometer dünnen Schichten, dessen Zweck darin liegt, Wellen
und ihre \textit{Interferenz} gezielt zu nutzen, um einen gewünschten
Reflexions- und Transmissionsgrad zu erreichen. Dünnschichtsysteme sind ein
wichtiger Bestandteil im Bereich der Optik und haben auch in anderen
Fachgebieten einige Anwendungen, z.B.:
\begin{itemize}
    \item Antireflexbeschichtungen in Brillengläsern, Hochreflexschichten wie in Spiegeln
          oder Kameraobjektiven in der Optik,
    \item Beschichtung von Geräten in der Medizintechnik,
    \item Herstellung von Prozessoren bei Mikroelektronik.
\end{itemize}
Dafür bedient man sich meist verschiedenster Materialien wie
z.B.Titanoxid, Magnesiumfluorid oder Aluminiumoxid, die alle einen
individuellen Brechungsindex besitzen.
Diese Eigenschaften werden in allen Anwendungen als das essenzielle
Werkzeug genutzt, um das jeweilige Ziel im gewünschten Wellenbereich zu erreichen.
Dies kann z.B. im Fall von Brillengläsern destruktive Interferenz sein, bei anderen
Anwendungen, wie Spiegeln, aber auch konstruktive Interferenz.

Im Folgenden wird die Berechnung des bereits besprochenen Reflexions- und
Transmissionsgrades beschrieben, um ein möglichst realistisches Programm zur
Erfassung eben dieser zu entwickeln. Sie wird in Abhängigkeit der vorliegenden
Brechungsindizes, Dicke der Schichten und dem Einfallswinkel berechnet.

Das größte Problem, das es hier zu lösen gilt, ist die Berechnung des
Reflexions- und Transmissionsgrades, da es durch die Existenz mehrerer
Schichten zu einem endlosen Prozess von Re-Kalkulationen durch Teilwellen
kommt, bei der bei jedem Durchdringen einer der Schichten ein neuer Prozess in
beide Richtungen gestartet wird, was die Berechnung auf üblichem Wege
erschwert. Eine Visualisierung des Prozesses in einem Mehrschichtsystem ist in
Abbildung 1 dargestellt. Um dieses Problem zu lösen wird die
Transfermatrixmethode angewandt, welche anhand der Brechungsindizes, des
Einfallswinkels und den Reflexions- und Transmissionskoeffizienten, an jeder
Grenzfläche, ein Reflexionsspektrum als Resultat liefert.

\begin{figure}[b]
    \centering
    \includegraphics[scale=0.2]{./assets/Grafik Reflexion.png}
    \caption{Darstellung der Reflexion und Transmission im Mehrschichtsystem}\label{fig:Reflexion}
\end{figure}

\section{Mathematisch-Physikalisches Modell}
\subsection{Brechungsindex}
Der (komplexe) Brechungsindex der jeweiligen Schicht hat die Form
\[\tilde{n} = n + i\kappa.\]
Sein realer Teil bestimmt die Brechung des Lichts an der Grenzfläche, während
sein imaginärer Teil die Absorption des Lichts beschreibt. Die
Wellenausbreitung kann durch den Term
\[e^{i\frac{n \omega}{c}z} \cdot e^{-\frac{\kappa\omega}{c}z}\]
beschrieben werden. Dabei ist die Absorption
\[e^{-\frac{\kappa\omega}{c}z}\]
und die Phasenverschiebung
\[e^{i\frac{n \omega}{c}z}.\]
Diese beiden Komponenten sind essenziell für spätere Berechnungen von
Schichteigenschaften mit der Transfermatrixmethode.

Des Weiteren ist es wichtig den Brechungsindex als Funktion der Wellenlänge
$n(\lambda)$ zu betrachten, um präzise Resultate zu erlangen, da Dispersion
berücksichtigt werden muss. Dafür wird im Laufe der Arbeit, wenn möglich, die
Sellmeier-Gleichung verwendet~\cite{10.1063/1.555616}:
\[n^2(\lambda) = 1 + \sum_{i} \frac{B_i \lambda^2}{\lambda^2 - C_i}.\]
Für manche Materialien werden zum Zweck eines präziseren Ergebnisses von der
Sellmeier-Gleichung abweichende Formeln oder Interpolation von
Messwertetabellen genutzt.~\cite{Kren:11}.
\subsection{Reflexion und Transmission}
Reflexions- und Transmissionskoeffizienten von elektromagnetischen Wellen an
einer Grenzfläche zwischen zwei Schichten werden mit den fresnelschen Formeln
ermittelt~\cite{Hecht+2018}. $\theta_1$ ist der Einfallswinkel des Lichts auf
die Grenzfläche und $\theta_2$ der Brechungswinkel. Der Brechungswinkel lässt
sich durch das snelliussche Brechungsgesetz definieren~\cite{Hecht+2018}:
\[n_1 \sin\theta_1 = n_2 \sin\theta_2.\]
Durch Auflösen nach $\theta_2$ entsteht folgende Gleichung:
\[\theta_2 = \arcsin(\frac{n_1}{n_2}\sin\theta_1).\]
Zusätzlich muss die Polarisation berücksichtigt werden. Abhängig davon, ob das
Licht senkrecht oder parallel zur Einfallsebene schwingt, variieren die
Formeln.

Senkrechte Polarisation:
\[r_s = \frac{n_1 \cos\theta_1 - n_2 \cos\theta_2}{n_1 \cos\theta_1 + n_2 \cos\theta_2}, \quad t_s = \frac{2n_1 \cos\theta_1}{n_1 \cos\theta_1 + n_2 \cos\theta_2}\]

Parallele Polarisation:
\[r_p = \frac{n_2 \cos\theta_1 - n_1 \cos\theta_2}{n_2 \cos\theta_1 + n_1 \cos\theta_2}, \quad t_p = \frac{2n_1 \cos\theta_1}{n_2 \cos\theta_1 + n_1 \cos\theta_2}\]
$n_1$ und $n_2$ stellen die Brechungsindizes der linken und rechten Schicht beim Grenzübergang dar.
Dieser Prozess liefert nun die Koeffizienten an der Grenzfläche zwischen zwei
Schichten und muss für jede Schicht, bzw. jeden Grenzübergang erfolgen. Die
Resultate werden in der Transfermatrixmethode verwertet.
\subsection{Die Transfermatrixmethode}
Es handelt sich bei der Transfermatrixmethode um eine elegante mathematische
Lösung für das Problem der unendlichen Teilwellen~\cite{Hecht+2018}. Anstatt
den Verlauf des Lichtstrahls zu verfolgen, wird das gesamte Mehrschichtsystem
betrachtet. Die Grundlage hierfür sind die fresnelschen Formeln, da sie das
Verhältnis der Amplituden auf der linken und rechten Seite der Grenzfläche
beschreiben und dadurch folgendes Gleichungssystem bilden:
\[R_{i+1}^{\prime} = R_i \cdot t_{i,i+1} + L_{i+1}^{\prime} \cdot r_{i+1,i}\]
\[L_i = R_i \cdot r_{i,i+1} + L_{i+1}^{\prime} \cdot t_{i+1,i}.\]
Das resultierende Gleichungssystem kann auch in der Matrixschreibweise
\[\begin{pmatrix} t_{i,i+1} & 0 \\ -r_{i,i+1} & 1 \end{pmatrix}
    \begin{pmatrix} R_i \\ L_i \end{pmatrix} =
    \begin{pmatrix} 1 & -r_{i+1,i} \\ 0 & t_{i+1,i} \end{pmatrix}
    \begin{pmatrix} R_{i+1}^{\prime} \\ L_{i+1}^{\prime} \end{pmatrix}\]
geschrieben werden, was das Muster für die Grenzflächen- und
Propagationsmatrizen liefert.

Es werden jeweils zwei Arten von Matrizen gebildet, dabei bestimmt die
Grenzflächenmatrix D die Reflexion und Transmission an der Grenzfläche, während
die Propagationsmatrix P die Phasenverschiebung innerhalb einer Schicht
beschreibt. Die Grenzflächenmatrix besitzt die Form
\[D = \frac{1}{t}\begin{pmatrix}
        1 & r \\
        r & 1
    \end{pmatrix},\]
wobei $r$ dem berechneten Reflexions- und $t$ dem Transmissionskoeffizienten
aus den fresnelschen Formeln entsprechen. Die Propagationsmatrix hat die Form
\[P = \begin{pmatrix}
        e^{-i \varphi} & 0             \\
        0              & e^{i \varphi}
    \end{pmatrix}\]
mit
\[\varphi = \frac{2 \pi n d \cos(\theta_2)}{\lambda},\]
wobei d die Dicke, n der Brechungsindex und $\theta_2$ der Brechungswinkel der
Schicht sind. Zuletzt bleibt die Transfermatrix selbst. Sie ergibt sich aus der
Multiplikation der entstandenen Matrizen
\[M = D_0 \cdot P_0 \cdot D_{1} \cdot P_{1} \dots D_{n-1} \cdot P_n \cdot D_n\]
und wird anschließend genutzt um den Reflexionsgrad R, nach der Definition der
Intensität einer elektromagnetischen Welle,
\[R = \left| \frac{M_{21}}{M_{11}} \right|^2\]
zu bestimmen.\clearpage
\section{Algorithmische Umsetzung}
Die mathematischen Beziehungen, die in Kapitel 2 beschrieben werden, finden in
einer modular strukturierten Softwarearchitektur Umsetzung. Diese teilt den
Berechnungsprozess in klar voneinander getrennte Komponenten auf. Das Programm
umfasst die Materialverwaltung, die Berechnung der Fresnel-Koeffizienten, die
konsistente Konstruktion der Transfermatrix und darauf basierende
Reflexionsberechnungen.

Alle Schritte sind durch eigene Funktionen oder Klassen voneinander abgegrenzt,
was das Programm wartbar und erweiterbar macht und es ermöglicht, dass es mit
beliebigen Schichtsystemen verwendet werden kann
\subsection{Materialmodellierung}

Zur Repräsentation einer Schicht dient die Klasse \texttt{Material}. Sie
speichert den Namen \texttt{name} des verwendeten Materials, eine Schichtdicke
\texttt{d} und einen Berechnungstyp \texttt{n\_type}. Dieser Berechnungstyp
entscheidet dann welche optionalen Parameter der Klasse festgelegt werden, die
später zur Berechnung von Brechungsindizes genutzt wird. Dabei handelt es sich
entweder um einen festen komplexen Brechungsindex \texttt{n}, die
Sellmeier-Koeffizient \texttt{B} und \texttt{C}, eine Formel in Form vom string
\texttt{formula} oder eine Messwerttabelle \texttt{table} als Dictionary.

Für die gegebene Wellenlänge $\lambda$ liefert sie den komplexen Brechungsindex
der entsprechenden Schicht. Je nach Material kommen verschiedene bewährte
Formeln zum Einsatz. Dadurch wird eine realistische Nachbildung der
Wellenlängenabhängigkeit möglich, ohne dass Anpassungen an den späteren
optischen Berechnungen erforderlich sind. Die Klasse ist zudem dafür zuständig,
die Materialparameter aus der JSON-Datei zu laden. Die Schichtfolge kann
flexibel über die Benutzeroberfläche festgelegt werden, ohne dass eine
Anpassung des Programmcodes erforderlich ist.
\subsection{Berechnung der Grenzflächenkoeffizienten}
Die Funktion
\begin{center}
    \texttt{fresenel\_coefficients(n1, n2, theta1, polarization)}
\end{center}
übernimmt die numerische Umsetzung der Fresnel-Gleichungen an einer Grenzfläche zwischen
zwei Schichten. Sie berechnet zuerst den Brechungswinkel $\theta_2$ mit dem snelliusschen Gesetz~\cite{Hecht+2018},
die Reflexions- und Transmissionskoeffizienten r und t und den für die Transfermatrix benötigten Phasenverlauf.
Die Funktion unterscheidet streng zwischen den beiden Polarisationen
\begin{center}
    \texttt{'Senkrecht'} (s-polarisiert) \quad und \quad
    \texttt{'Parallel'} (p-polarisiert)
\end{center}
und gibt die zugehörigen Koeffizienten zurück.
Damit ist die Polarisation vollständig als Parameter steuerbar, was eine Simulation beider
physikalischen Fälle ermöglicht.\clearpage
\subsection{Aufbau der Transfermatrix}
Die Funktion \begin{center} \small \texttt{transfer\_matrix(material\_list, d\_list,
        wavelength, polarization, theta0)}
\end{center}
durchläuft alle Schichten und führt für jede Grenzfläche die
bekannten Schritte aus dem mathematisch-physikalischen Modell aus. Sie
ermittelt die fresnelschen Koeffizienten aus den Brechungsindizes der
angrenzenden Schichten, wobei für die Brechungsindizes für jede einzelne
Wellenlänge die Material-Methode
\begin{center}
    \texttt{reflective\_index(self, wavelength)}
\end{center}
genutzt wird. Daraufhin wird mit den Reflexions- und Transmissionskoeffizienten
die Grenzflächenmatrix D generiert und anschließend für alle endlichen
Schichten in Übereinstimmung mit der Dicke, dem Einfallswinkel und dem
Brechungsindex, auch die Propagationsmatrix P. Zuletzt werden beide Matrizen
multiplikativ in die Gesamtmatrix eingebaut. Die Transfermatrix wird iterativ
aufgebaut durch
\[M \leftarrow M \cdot D \cdot P,\]
was sicherstellt, dass der aktuelle optische Zustand des Lichtfelds im
Mehrschichtsystem erhalten bleibt. Dabei werden Schichten mit unendlicher Dicke
(Luft und Substrat) automatisch identifiziert, und ihnen wird keine
Propagationsmatrix zugewiesen. So wird das physikalische Modell exakt
wiedergegeben
\subsection{Berechnung des Reflexionsgrades}
Die Funktion
\begin{center}
    \texttt{reflectance(material\_list, wavelengths, polarization, theta)}
\end{center}
fasst alle vorher beschriebenen Schritte zu einer vollständigen Simulation
zusammen. Sie übernimmt, die Umrechnung der Schichtdicken in SI-Einheiten,
berechnet beliebige spektrale Abtastungen, berücksichtigt dabei mehrere oder
einen Einfallswinkel, ruft die Transfermatrix auf und kümmert sich zuletzt um
die Herausarbeitung des Reflexionskoeffizienten.
Der Amplitudenreflexionskoeffizient wird aus der Transfermatrix bestimmt:
\begin{center}
    \texttt{r = M[1,0] / M[0,0]}
\end{center}
woraus letztlich der Reflexionsgrad resultiert:
\[R = |r|^2.\]
Weil die Berechnung vollständig vektorisiert ist, können ganze Winkel- oder
Wellenlängenbereiche mit nur einem Funktionsaufruf evaluiert werden, ohne dass
der Nutzer zusätzliche Einstellungen vornehmen muss.
\section{Benutzerhandbuch}
Hier wird die grafische Nutzeroberfläche der Software präsentiert, zusammen mit
einzelnen Funktionen, denen es eine Erläuterung bedarf.

Zu aller erst müssen die Abhängigkeiten installiert werden. Da die Software auf
Python basiert, kann dies am einfachsten über das Paketverwaltungsprogramm pip
getätigt werden. In der Repository ist eine \texttt{requirements.txt} dafür zu
finden, welche über pip mit dem Befehl
\begin{center}
    \texttt{pip install -r requirements.txt}
\end{center}
zu installieren sind. Beachtet werden sollte hierbei, dass eine globale
Installation auf Linux Distributionen nicht auf denselben Weg möglich ist.
Hierfür sollte vorher mit
\begin{center}
    \texttt{python -m venv .venv}
\end{center}
eine virtuelle Umgebung für das Programm erstellt werden. Die Installation ist
daraufhin nach Aktivieren dieser virtuellen Umgebung wie vorher beschrieben
auszuführen. Anschließend wird die Software nach erfolgreicher Installation mit

\begin{center}
    \texttt{Python GUI.py}
\end{center}
ausgeführt.

Nun sollte eine GUI zu sehen sein, welche aus einer Tabelle, einem Graphen und
einzelnen zu bestimmenden Parametern unter beiden Flächen besteht. Dies ist
zusätzlich in Abbildung 2 zusehen.
\begin{figure}[htbp]
    \centering
    \includegraphics[scale=0.45]{./assets/GUI.png}
    \caption{Beschreibung der Benutzeroberfläche}\label{fig:GUI}
\end{figure}
Es stehen in der Tabelle bereits 2 Zeilen vor angefertigt zur Verfügung. Sie
stellen in der Software das Einfallsmedium und Substrat dar und können
entsprechend nicht entfernt werden. Am Einfallsmedium befindet sich in der
letzten Spalte ein Button, welcher eine weitere Zeile, bzw. Schicht, direkt
nach ihr selbst generiert und einfügt. Alle Zeilen, die eine Schicht
repräsentieren, welche weder das Einfallsmedium, noch das Substrat sind,
besitzen zwei Buttons in ihrer letzten Spalte. Das Pluszeichen hat dieselbe
Funktion, wie beim Einfallsmedium, während das Minuszeichen die Schicht, in der
sie beinhaltet ist, aus der Simulation löscht.

In der ersten Spalte unter \texttt{Material} ist ein Dropdown-Menü mit der
Aufschrift Presets zu sehen. Benutzer können hier eine Auswahl von 5
vorgefertigten Schichten wählen, welche bereits alle Parameter belegt haben. Es
steht dem Nutzer jedoch offen die Dicke der Schicht anzupassen. Alternativ
besteht auch die Möglichkeit mit dem Button unter der Tabelle ein neues
Material anzulegen. Das Betätigen öffnet ein zusätzliches Fenster, in dem der
Nutzer einen Namen für das Material und einen komplexen Brechungsindex
festlegen kann. Das Bestätigen der Eingabe erstellt ein neues Materialobjekt
und fügt dessen Daten in das \texttt{Material.json} im Root ein, damit sie
zusammen mit den vorgefertigten Materialien für den Nutzer langzeitig
gespeichert sind. In Abbildung 3 ist das Dialogfenster abgebildet.
\begin{figure}[htbp]
    \centering
    \includegraphics[scale=0.6]{./assets/neu.png}
    \caption{Dialogfenster zum Hinzufügen neuer Materialien}
    \label{fig:GUI-neu}
\end{figure}

\texttt{Bestätigen} plottet den Reflexionsgrad in Abhängigkeit
der gegebenen Parameter in den Graphen. Wichtig ist hierbei die Definition des
Wellenlängenbereichs und des Einfallswinkels, da sich die berechnete Funktion
sich je nach Angabe ändert. Hierbei muss eins der folgenden Muster eingehalten
werden:
\begin{align*}
    (\text{\texttt{1.Wellenlänge - 2.Wellenlänge}, \quad \texttt{Einfallswinkel}})    & \rightarrow R(\lambda) \\
    (\text{\texttt{Wellenlänge}, \quad \texttt{1.Einfallswinkel - 2.Einfallswinkel}}) & \rightarrow R(\theta)
\end{align*}

Sollten die Parameter in einem ungültigen Bereich liegen, wird stattdessen eine
Fehlermeldung geliefert. Weiteres Nutzen des Bestätigen-Buttons fügt die
gewünschten Funktionen in den Graphen hinzu, bis der Nutzer sich entscheidet,
diesen mit \texttt{Zurücksetzen} zu leeren.

Eine Übersicht zu den Parametern und validen Eingaben befindet sich in Tabelle
1.
\begin{table}[htbp]
    \centering
    \caption{Beschreibung der Eingabeparameter im Programm}\label{tab:Parameter}
    \begin{tabular}{|l|p{9cm}|}
        \hline
        Parameter           & Beschreibung                                                             \\
        \hline
        Dicke               & Dicke der entsprechenden Schicht in Nanometern                           \\
        \hline
        n-real              & Realer Teil des (komplexen) Brechungsindex der
        entsprechenden Schicht                                                                         \\
        \hline
        n-imaginärer        & Imaginärer Teil des (komplexen) Brechungsindex
        der entsprechenden Schicht                                                                     \\
        \hline
        Wellenlängenbereich & Gewünschter Wellenlängenbereich $>0$, der im Graphen
        geplottet werden soll. Kann angegeben werden in der Form: \texttt{1.Wellenlänge-2.Wellenlänge} \\
        \hline
        Einfallswinkel      & Einfallswinkel in Gradmaß im Bereich $0^\circ-89^\circ$                  \\
        \hline
        Polarisation        & Art der Polarisation mit Auswahl zwischen senkrecht und parallel         \\
        \hline
    \end{tabular}
\end{table}

\section{Simulationsergebnisse}

In diesem Kapitel werden zwei der Extremen unserer Anwendung beobachtet. Zuerst
wird eine Antireflexbeschichtung betrachtet. Sie ist aufgeteilt in die
jeweiligen Materialien und ihren individuellen Dicken, welche in Tabelle 2 zu
sehen sind und soll im sichtbaren Spektrum simuliert werden. Hierfür nehmen wir
einen Wellenlängenbereich von 400 nm bis 800 nm mit einem senkrechten
Einfallswinkel.
\begin{table}[h]
    \centering
    \caption{Zu simulierendes Schichtsystem}\label{tab:Angabe1}
    \begin{tabular}{|l |c|}
        \hline
        Material    & Dicke in nm \\
        \hline
        Luft        & $\infty$    \\
        \hline
        MgF$_2$     & 102         \\
        TiO$_2$     & 105         \\
        AL$_2$O$_3$ & 79          \\
        \hline
        Glas        & $\infty$    \\
        \hline
    \end{tabular}
\end{table}
Das Ergebnis ist in Abbildung 4 dargestellt und zeigt keine physikalischen
Auffälligkeiten, die auf einen Fehler deuten würden. Die Funktion ist im
ausgewählten Bereich stetig, was auf mathematische Stabilität deutet.
Zusätzlich kann der Reflexionsgrad mit dem eines Lichtstrahls der von Luft auf
bloßes Glas trifft, im Graph verglichen werden.

Die zweite zu simulierende Schicht befindet sich in Tabelle 3. Betrachtet wird
hierbei ein Hochreflektor im EUV-Bereich. Es wurde versucht für dieses
Schichtsystem bei extremem Ultraviolettlicht eine passende Dicke der Schichten
zu ermitteln, um optimale Reflexivität zu garantieren.
\begin{table}[htpb]
    \centering
    \caption{Zu simulierendes EUV-Schichtsystem}\label{tab:Angabe2}
    \begin{tabular}{|l |c|}
        \hline
        Material & Brechzahl bei 13.5 nm                                 \\
        \hline
        Mo       & $1 - 7,6044 \cdot 10^{-2} -  6,4100 \cdot 10^{-3}i $  \\
        \hline
        Si       & $1 -  9,3781 \cdot 10^{-4} - 1,7260 \cdot 10^{-3}i  $ \\
        \hline
        SiO$_2$  & $1- 0,0218338 - 0,010721i$                            \\
        \hline
    \end{tabular}
\end{table}
\begin{table}[htbp]
    \centering
    \caption{Optimierte Schichtdicke für EUV-Schicht}\label{tab:Schicktdicke}
    \begin{tabular}{|l |c| c|}
        \hline
        Material & optimierte Dicke für 0$^\circ$ \\
        \hline
        Mo       & $2,8nm$                        \\
        \hline
        Si       & $4,1nm$                        \\
        \hline
    \end{tabular}
\end{table}
\begin{figure}[htpb]
    \centering
    \includegraphics[scale=0.5]{./assets/1 Ergebnis Senkrecht.png}
    \caption{1. Simulationsergebnis für $R(\lambda)$ bei senkrechtem Einfall}\label{fig:Ergebnis1-0}
\end{figure}
Die idealen Werte für den senkrechten Einfall sind in Tabelle 4 zu sehen.
Die Ergebnisse in Abbildung 5 ist die gewünschte Funktion für die EUV-Schicht
mit ihren optimierten Dicken. Der Hochpunkt bei etwa 13.5 nm bestätigt die
gewählten Werte der Schichtdicke als optimal für die gewünschte Anwendung.
\begin{figure}[h!]
    \centering
    \includegraphics[scale=0.6]{./assets/Hochreflektor EUV.png}
    \caption{Darstellung von $R(\lambda)$ für die EUV-Schicht bei Senkrechtem Einfallswinkel}\label{fig:Ergebnis2-0}
\end{figure}
\clearpage
\printbibliography[]
\end{document}